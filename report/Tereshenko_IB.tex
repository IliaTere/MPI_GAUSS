\documentclass[12pt, a4paper]{article}
\usepackage{amsmath}
\usepackage{amssymb}
\usepackage{times}
\usepackage{mathptmx}
\usepackage[utf8]{inputenc}
\usepackage{listings}
\usepackage{xcolor}
\usepackage[english,russian]{babel}
\usepackage[T2A]{fontenc}


\definecolor{codegray}{rgb}{0.5,0.5,0.5}
\definecolor{backcolour}{rgb}{0.95,0.95,0.92}
\definecolor{TRE} {rgb}{0.9,0.3,0.5}

\lstset { %
    language=C++,
    backgroundcolor=\color{black!5}, 
    basicstyle=\footnotesize,
}

\usepackage[
  a4paper, mag=1000, includefoot,
  left=1.1cm, right=1.1cm, top=1.2cm, bottom=1.2cm, headsep=0.8cm, footskip=0.8cm
]{geometry}



\mathsurround=0.1em
\clubpenalty=1000%
\widowpenalty=1000%
\brokenpenalty=2000%
\frenchspacing%
\tolerance=2500%
\hbadness=1500%
\vbadness=1500%
\doublehyphendemerits=50000%
\finalhyphendemerits=25000%
\adjdemerits=50000%

\title{Отчет по анализу блочного алгоритма нахождения обратной матрицы методом Гаусса с выбором элемента по столбцу(MPI)}
\author{Терешенко Илья}

\begin{document}

\maketitle

\section{Постановка задачи}
Найти для матрицы A = $\begin{pmatrix}
     a_{11}& a_{12} &\ldots & a_{1n}\\
     a_{21}& a_{22} &\ldots & a_{2n}\\
     \vdots& \vdots &\ddots & \vdots\\
     a_{n1}& a_{n2} &\ldots & a_{nn}    
    \end{pmatrix}$ обратную матрицу методом Гаусса с выбором главного элемента по столбцу

\section{Разделение данных на свои и чужие}
Своими данными для каждого процесса будем считать блочные строки матрицы, занумерованные следующим образом A = $\begin{pmatrix}
     A_{11}& A_{12} &\ldots & A_{1n} | 1\\
     \hline
     A_{21}& A_{22} &\ldots & A_{2n}| 2\\
     \hline
     \vdots& \vdots && \vdots\\
     \hline
     A_{p1}& A_{p2} &\ldots & A_{pn}|p\\
     \hline
     A_{(p+1)1}& A_{(p+1)2} &\ldots & A_{(p+1)n}|1\\
     \hline
      A_{(p+2)1}& A_{(p+2)2} &\ldots & A_{(p+2)n}|2\\
     \vdots& \vdots &\ddots & \vdots\\
     A_{n1}& A_{n2} &\ldots & A_{nn}    
    \end{pmatrix}$, где p это количество потоков. Аналогично для присоединенной матрицы. Каждый процесс имеет доступ к своим данным и не имеет доступа к чужим. Для того что заполнять матрицу, а также делать различные преобразования над строками между процессами необходимо уметь переходить от глобальной нумерации к локальной. Для этого будет использовать следующие функции.
    \begin{lstlisting}[language=C++,
tabsize=2,
stepnumber=1,
numbers=left,
 commentstyle=\itshape \color{TRE},
backgroundcolor=\color{backcolour},
basicstyle=\ttfamily\small,
frame=single,
breaklines=true,   
breakatwhitespace=true,
emphstyle=\itshape,
keywordstyle=\color{blue} ]
int g2l(int n, int m, int p, int k, int i_glob)
{
  int i_glob_m = i_glob/m;
  int i_loc_m = i_glob_m/p;
  return i_loc_m*m+i_glob%m;
}
\end{lstlisting}
\begin{lstlisting}[language=C++,
tabsize=2,
stepnumber=1,
numbers=left,
 commentstyle=\itshape \color{TRE},
backgroundcolor=\color{backcolour},
basicstyle=\ttfamily\small,
frame=single,
breaklines=true,   
breakatwhitespace=true,
emphstyle=\itshape,
keywordstyle=\color{blue} ]
int l2g(int n, int m, int p, int k, int i_loc)
{
  int i_loc_m = i_loc/m;
  int i_global_m = i_loc_m*p + k;
  return i_global_m*m+i_loc%m;
}
\end{lstlisting}

\section {Описание алгоритма}
g - номер процесса, l = n\%m, k=n/m, bl = (l==0?k:k+1), p - количество процессов
\begin{enumerate}
    \item На шаге с номером step каждый процесс, который хранит строку с номером $\geq step$ ( в глобальной нумерации) находит норму обратной матрицы для блока с номером $(i, step),\ i\in \{g+jp\}\ j =1\dots bl$ (в глобальной нумерации) в локальной нумерации необходимо найти обратную матрицу к блоку с указателем на начало = $(step-1)m^2$, т.e к блоку $(0, step)$.
    \item С помощю \text{MPI\_Allreduce} процессы находят блок с наименьшей нормой обратного среди всех, если такого блока не найдено, то алгоритм не применим.
    \item Пусть такой блок найден и его номер t, тогда \text{MPI\_Bcast} блока с наименьшей нормой обратной всем остальным от процесса, который его нашел. С помощью \text{MPI\_sendrecv} обмениваются строками владелец строки с номером step и владелец строки с номером t. На этом шаге размер отправляемых данных: $m*m + 2m*n$
    \item Каждый процесс домножает свои блоки в буффере на полученный блок слева\\
    $(Block)^{-1}  Buf_{0i}, i\in \{step+1,\ldots, bl\} $\\
    $(Block)^{-1}  Buf1_{0i}, i\in \{0,\ldots, bl\}$ Buf1 - строка от присоединенной матрицы.\\ После чего вычитает из своих блочный строк по таким формулам: \\
    $A_{ij} = A_{ij} - Buf_{0j}*A_{i,step};\ j \in \{step+1,\dots, bl\} , i\in \{0,\dots, rows\}$\\
    $ rows = (n+m-1)/m\%p=k?b/p:b/(p+1)$\\
    $B_{ij} = B_{ij}-Buf1_{0j}*A_{i,step}; j\in \{0,\dots , bl\};\ i\in \{0,\dots, rows\}$
    \item На шаге с номером k+1 процесс, который владет k+1-ой строкой в глобальной нумерации находит обратную матрицу для блока $A_{rows, rows}$(в локальной нумерации) и выполняет следующие \\
    $B_{rows, i} = A^{-1}*B_{rows, i}; i\in \{0, \dots , rows\} $
    \item Обратный ход. Находим владельца текущей строки $owner = step\%p$, который отправит \text{MPI\_Bcast} блока 
    с локальными координатами $i=step/p*m + step*m\%m;A_{(i, step)}=A$, а также строку из приписанной матрицы B. Таким образом объем отправленных данных $m*m+n*m$.\\
    Все процессы работают по таким формулам: \\
    $B_{ji} = B_{ji} - B^{owner}_{step,i}*A_{step-1, step};\ где j\in \{step-1,\dots, 0\}; i\in \{0,\dots,bl\}$
\end{enumerate}


\section{Нахождение сложности}

В данном разделе будет использован факт о том, что сложность нахождения обратной матрицы размера $n \times n$ неблочным методом Гаусса составляет $\frac{8}{3} \times n^3 + O(n^2)$ и сложность умножения двух матриц того же размера $2n^3 - n^2$. Распишем по пунктам сложность групп операций алгоритма:
\begin{enumerate}
\item Для этого в первом столбце посчитаем обратные матрицы в количестве k штук, во втором k-1 и так далее до k столбца, где необходимо будет обратить 1 матрицу(позже на обратные матрицы выбранных элементов мы будем умножать соответствующие матричные строки). Получаем сложность(здесь и далее полагаем $k = \frac{n}{m}$ целым):
\begin{equation}
\begin{aligned}
D=(g+1,g+1+p, ...,k)(\frac{8}{3}m^3 + O(m^2)) \times \sum_{i\in D}^{}i = (\frac{8}{3} m^3 + O(m^2)) \times \frac{(k+1)k}{2p} = (\frac{8}{3} m^3 + O(m^2)) \times \frac{(\frac{n}{m}+1)\frac{n}{m}}{2p} = \\ \frac{4}{3p}n^2m + \frac{4}{3p}nm^2 + O(n^2 + nm)
\end{aligned}
\end{equation} 
\item В каждой строке исходной матрицы необходимо будет умножать по формуле (18) определенное количество элементов на обратную матрицу, найденную на соответсвующем шаге. Получаем сложность:
\begin{equation}
\begin{aligned}
D=(g+1,g+1+p, ...,k-1)(2m^3 - m^2) \times \sum_{i \in D}^{}i = (2m^3 - m^2) \times \frac{k(k-1)}{2p} = (2m^3 - m^2) \times \frac{\frac{n}{m}(\frac{n}{m}-1)}{2p} = \\ \frac{n^2m}{p} - \frac{nm^2}{p} + O(n^2 + nm)
\end{aligned}
\end{equation}
\item При занулении на step шаге необходимых элементов исходной матрицы по формуле (9) понадобится мультипликативных операций:
\begin{equation}
\begin{aligned}
D=(g+1,g+1+p, ...,k-1)(2m^3 - m^2) \times \sum_{i \in D}^{}i^2 = (2m^3 - m^2) \times \frac{\frac{k}{p}(k-1)(2k+1)}{6} =\\ (2m^3 - m^2) \times \frac{\frac{n}{m}(\frac{n}{m}-1)(2\frac{n}{m}+1)}{6p} = \frac{2}{3p}n^3 - \frac{1}{3p}n^2m - \frac{1}{3p}nm^2 + O(n^2 + nm)
\end{aligned}
\end{equation}
\item При занулении на p шаге необходимых элементов исходной матрицы по формуле (9) понадобится аддитивных операций(учитывая, что для вычитания из матрицы m на m матрицы того же размера необходимо $m^2$ операций):
\begin{equation}
\begin{aligned}
D=(g+1,g+1+p, ...,k-1) |m^2 \times \sum_{i \in D}i^2 = m^2 \times \frac{\frac{k}{p}(k-1)(2k+1)}{6} = m^2 \times \frac{\frac{n}{mp}(\frac{n}{m}-1)(2\frac{n}{m}+1)}{6} = \\ \frac{n^3}{3mp} - \frac{n^2}{6p} - \frac{nm}{6p} = O(n^2 + nm)
\end{aligned}
\end{equation}
\item Теперь обратимся к присоединенной матрице. Сложность операций умножения на обратные матрицы:
\begin{equation}
\frac{k^2}{p} \times (2m^3 - m^2) = \frac{n^2}{m^2p} \times (2m^3 - m^2) =\frac{1}{p} (2n^2m - n^2) = \frac{2n^2m}{p} + O(n^2)
\end{equation}
\item Для умножения блоков присоединенной матрицы при вычислениях по формуле (26) понадобится мультипликативных операций:
\begin{equation}
\begin{aligned}
D=(g+1,g+1+p, ...,n/m-1)\\(2m^3 - m^2) \times k \times \sum_{i \in D}i = (2m^3 - m^2) \times \frac{n}{m} \times \sum_{i \in D}i = \\ (2m^3 - m^2) \times \frac{n}{m} \times \frac{\frac{n}{mp}(\frac{n}{m}-1)}{2} = \frac{n^3}{p} - \frac{n^2m}{p} +O(n^2)
\end{aligned}
\end{equation}
\item Для вычитания блоков присоединенной матрицы при вычислениях по формуле (26) понадобится аддитивных операций:
\begin{equation}
\begin{aligned}
D=(g+1,g+1+p, ...,\frac{n}{m}-1)\\m^2 \times k \times \sum_{i \in D}i = m^2 \times \frac{n}{m} \times \sum_{i \in D}i = \\ m^2 \times \frac{n}{m} \times \frac{\frac{n}{mp}(\frac{n}{m}-1)}{2} = O(n^2)
\end{aligned}
\end{equation}
\item Осталось рассмотреть обратный ход. Исходя из формулы (38) понадобится мультипликативных операций:
\begin{equation}
\begin{aligned}
D=(g+1,g+1+p, ...,k-1)\\(2m^3 - m^2) \times k \times \sum_{i \in D}i = (2m^3 - m^2) \times \frac{n}{m} \times \sum_{i \in D}i = \\ (2m^3 - m^2) \times \frac{n}{m} \times \frac{\frac{n}{mp}(\frac{n}{m}-1)}{2} = \frac{n^3}{p} - \frac{n^2m}{p} +O(n^2)
\end{aligned}
\end{equation}
\item Из формулы (38) аддитивных операций на обратный ход необходимо:
\begin{equation}
\begin{aligned}
D=(g+1,g+1+p, ...,k-1)\\m^2 \times k \times \sum_{i \in D}i = m^2 \times \frac{n}{m} \times \sum_{i \in D}i = \\ m^2 \times \frac{n}{m} \times \frac{\frac{n}{mp}(\frac{n}{m}-1)}{2} = O(n^2)
\end{aligned}
\end{equation}
\item Итого сложность:
\begin{equation}
S(n,m, p) = \frac{8}{3p}n^3 + \frac{3}{p}n^2m - \frac{nm^2}{p} + O(n^2 + m^2)
\end{equation}
\end{enumerate}

\section{Сложность алгоритма в краевых случаях}
\begin{equation}
S(n,1, 1) = \frac{8}{3}n^3 + O(n^2)
\end{equation}

\begin{equation}
S(n,n, p) = \frac{14}{3p}n^3 + O(n^2)
\end{equation}
\section{Оценка числа обменов}
В процессе точки коммуникации вызываются на каждом шаге алгоритма 6 раз 
\begin{equation}
\sum_{i=1}^{n/m}6=6\frac{n}{m} = O(n)
\end{equation}
\section{Оценка объёма обменов}
\begin{equation}
    \sum_{i=1}^{\frac{n}{mp}}2(m^2+2mn) = \frac{n}{pm}2m^2 + 4\frac{n}{pm}mn=\frac{2nm}{p} + \frac{4}{p}n^2
\end{equation}
\end{document} 
